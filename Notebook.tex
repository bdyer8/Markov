%!TEX TS-program = xelatex
%!TEX encoding = UTF-8 Unicode

\documentclass[12pt]{article}
\usepackage[letterpaper, margin=1in]{geometry}
\usepackage{graphicx}
\usepackage{amssymb}
\usepackage{lastpage}
\usepackage{fancyhdr}
\usepackage{fontspec,xltxtra,xunicode}
\defaultfontfeatures{Mapping=tex-text}
\setromanfont[Mapping=tex-text]{Times}
\setsansfont[Scale=MatchLowercase,Mapping=tex-text]{Helvetica}
\setmonofont[Scale=MatchLowercase]{Andale Mono}

\pagestyle{fancy}
\renewcommand{\footrulewidth}{0pt}
\lhead{Blake Dyer}
\rhead{\thepage~of \pageref{LastPage}}
\chead{Theoretical Sedimentology: Carbonates}
\fancyfoot{}

\title{\sffamily \Huge Theoretical Sedimentology: Carbonates}
\author{\sffamily Blake Dyer, Princeton University}
\date{}                                           % Activate to display a given date or no date

\begin{document}
\maketitle
\thispagestyle{empty}
\vspace{6em}

%\begin{center}
%\includegraphics[width=.99\textwidth]{stratSummary2.pdf}   
%\end{center}

\newpage

\setcounter{page}{1}
\section*{\sffamily 02 June 2015}
Today I managed to get python - goal working with DEM files from the bahamas.  the GMRT dataset is pretty good around Andros island, low resolution to the north.  I might be able to find another DEM and patch that, not sure.  For now, I will continue with developing some python scripts to work with the DEM data and geopandas shapefiles of facies maps for the region.  then, i will start working on developing the walkers for generating transition matrices.  take a look at [harris2015] for the water energy information.  also, add some figures from today

%\begin{center}
%\includegraphics[width=.99\textwidth]{DEMspline.pdf}   
%\end{center}

\end{document}  